\documentclass[12pt]{article}

% Packages
\usepackage[utf8]{inputenc}
\usepackage{amsmath, amssymb, amsfonts}
\usepackage{graphicx}
\usepackage{float}
\usepackage{geometry}
\usepackage{booktabs}
\usepackage{hyperref}
\usepackage{mathtools}
\usepackage{lmodern}
\usepackage{setspace}
\usepackage{titlesec}
\usepackage{caption}

% Page layout
\geometry{a4paper, margin=1in}
\setstretch{1.3}
\titleformat{\section}{\normalfont\Large\bfseries}{\thesection}{1em}{}
\titleformat{\subsection}{\normalfont\large\bfseries}{\thesubsection}{1em}{}
\hypersetup{
    colorlinks=true,
    linkcolor=black,
    citecolor=black,
    urlcolor=blue
}

% Title (optional)
\title{\textbf{pyvar LLM Knowledge Base}}
\author{}
\date{}

\begin{document}

\maketitle
\tableofcontents
\newpage

\section{Introduction}

We present the mathematical foundations underlying each of our models. We begin with single-asset models, also referred to as univariate models, which focus on the return dynamics of a single asset or of a portfolio treated as a single aggregated position.

We then introduce a range of portfolio-level models, which explicitly consider the joint distribution of multiple asset returns. These include both analytical and simulation-based approaches. The section concludes with the backtesting methods used to evaluate the accuracy and reliability of our risk estimates.


\section{Basic VaR Models}

Value-at-Risk (VaR) at tail probability level $\alpha$ is defined as the smallest threshold $z_{\alpha}$ such that the probability of a loss exceeding it is $\alpha$. Equivalently, it is the $(1 - \alpha)$ quantile of the return distribution:
\[
\Pr\left(r_t < -z_{\alpha}\right) = \alpha.
\]
In monetary terms, this is typically scaled by \( W \), where \( W \) denotes the portfolio wealth. For simplicity, in single-asset models we omit \( W \) and use percentage-based notation.

We start with models that assume constant volatility over time. In the non-parametric (historical) approach:
\[
\mathrm{VaR}_{\alpha}^{\mathrm{hist}} = -\,\widehat{Q}_{\alpha}(r_t), \qquad
\mathrm{VaR}_{\alpha}^{\mathrm{normal}} = z_{\alpha} \sigma,
\]
and for portfolios this naturally extends to the normal parametric form. Expected Shortfall (ES), also known as Conditional VaR, is computed as:
\[
\begin{aligned}
\mathrm{ES}_{\mathrm{hist}} &= -\mathbb{E}\left[r_t \mid r_t < -z_{\alpha}\right], &
\mathrm{ES}_{\mathrm{normal}} &= \sigma \cdot \frac{\phi(z_{\alpha})}{\alpha}, &
\mathrm{ES}_{t_\nu} &= \sigma \cdot \frac{f_{t_\nu}(z_{\alpha})}{\alpha} \cdot \frac{\nu + z_{\alpha}^2}{\nu - 1},
\end{aligned}
\]
where $\phi$ is the standard normal density and $f_{t_\nu}$ is the density of the Student-$t$ distribution with $\nu$ degrees of freedom.

Normal and Student-$t$ VaR and ES estimates scale with the holding period as $\sqrt{h}$. If a longer horizon is desired for historical VaR, one can simply use historical data aggregated over the corresponding horizon.

Despite their simplicity, these models have important drawbacks. The parametric normal VaR often underestimates risk, as daily financial returns typically exhibit fat tails. The Student-$t$ model improves tail fit but cannot capture the asymmetry of return distributions. The GED is sometimes used as an alternative, but we do not implement it here due to its overlap with skew-$t$, which is also not included.

Historical methods are the most widely used among basic VaR models due to their simplicity and minimal assumptions. Parametric models usually assume i.i.d.\ returns, which limits their realism—especially since financial markets exhibit time-varying volatility. Still, parametric normal VaR remains popular thanks to its closed-form ES expression and ease of scaling by time horizon.



\section{Extreme Value Theory}

Technically still a parametric model, EVT is more sophisticated than the methods discussed above. We implement it with the Peaks-Over-Threshold (POT) approach: once a high threshold $u$ (e.g., the 99th-percentile loss) is chosen, a Generalized Pareto Distribution (GPD) is fitted to the excesses above $u$. 

VaR and ES are then derived analytically from the fitted GPD parameters. This allows for more accurate estimation of rare, extreme losses. The formulas used are:

\[
\begin{aligned}
{\text{VaR}}_\alpha &= u + \frac{\hat{\beta}}{\hat{\xi}}
\Bigl[\bigl(\tfrac{n}{n_u}(1-\alpha)\bigr)^{-\hat{\xi}} - 1\Bigr],\qquad
{\text{ES}}_\alpha &= \frac{{\text{VaR}}_\alpha + \hat{\beta} - \hat{\xi} u}{1 - \hat{\xi}},
\end{aligned}
\]

where $\hat{\xi}$ and $\hat{\beta}$ are the estimated shape and scale parameters, $n$ is the sample size, and $n_u$ is the number of exceedances.

Among parametric methods, EVT offers significantly greater precision in the estimation of tail risk. It naturally accommodates very high confidence levels and is based on solid asymptotic statistical theory, unlike the normal distribution which is largely ad hoc and unrealistic for financial losses.

However, EVT models are more complex to estimate and rely heavily on sufficient data in the tail. The results are highly sensitive to the chosen threshold $u$, and poor threshold selection can compromise accuracy. As such, EVT is most suitable when modeling extreme risks with a sufficiently large dataset.


\section{Volatility Models}

We have implemented the most popular univariate volatility models.

The GARCH(1,1) model is specified as:
\[
  \sigma_t^2 = \omega + a\, r_{t-1}^2 + b\, \sigma_{t-1}^2, \qquad r_t = \sigma_t z_t,
\]
where \( z_t \) are i.i.d.\ standardized shocks. VaR is then computed as:
\[
  \text{VaR}_{t,\alpha} = -\hat{\sigma}_t z_\alpha.
\]

Future volatility can be forecasted analytically using GARCH(1,1).

The \( \tau \)-step ahead forecast and cumulative \( T \)-step forecast are:
\begin{align*}
  \mathbb{E}[\sigma_{t+\tau}^2] &= \mathrm{VL} + (a + b)^\tau(\sigma_t^2 - \mathrm{VL}),  \\  
  \mathbb{E}[\sigma_{t,T}^2] &= \mathrm{VL}\left(T - 1 - \frac{(a + b)(1 - (a + b)^{T - 1})}{1 - (a + b)}\right)
    + \sigma_t^2\, \frac{1 - (a + b)^T}{1 - (a + b)},
\end{align*}
which naturally allow multi-period VaR estimation.

These models—especially GARCH and its extensions—are among the best performing in practice for financial data. They are designed to capture the volatility clustering, leptokurtosis, and persistence seen in asset returns.

We also support:

- EGARCH, which models the log of variance and captures asymmetric responses (leverage effects) of volatility to shocks.

- GJR-GARCH, which adds an indicator for negative returns to allow higher volatility responses to losses than to gains.

- APARCH, which introduces a power term in the volatility equation, allowing flexible tail behavior and better control over asymmetry.

All models are estimated by maximum likelihood using different innovation distributions: Normal, Student-\( t \), Generalized Error Distribution (GED), and Skew-\( t \). The choice of innovation distribution can significantly improve tail risk estimates.

For simpler benchmarks we include ARCH(\( p \)) and two rolling estimators, Moving Average (MA) and Exponentially Weighted Moving Average (EWMA):

\[
\sigma_{t}^{2,\mathrm{ARCH}}=\omega+\sum_{i=1}^{p} \alpha_{i}\, r_{t-i}^{2},\qquad
\sigma_{t}^{2,\mathrm{MA}}=\frac{1}{n}\sum_{i=1}^{n} r_{t-i}^{2},\qquad
\sigma_{t}^{2,\mathrm{EWMA}}=\lambda \sigma_{t-1}^{2}+(1-\lambda) r_{t-1}^{2}.
\]

ES for volatility models becomes:
\[
  \text{ES}_{t,\alpha} = -\hat{\sigma}_t \, \mathbb{E}[z_t \mid z_t < z_\alpha].
\] 

All our volatility models use a semi-empirical approach for the VaR computation, first calculating the empirical distribution of innovations with their respective volatility estimate and then using the required percentile of those innovations as \( z \).

Despite being slightly more complex than basic models, volatility models offer a major improvement by capturing the time-varying nature of financial risk.


\section{Time-Varying Correlation Models}

We now introduce models designed to capture time-varying correlations at the portfolio level, extending beyond single assets. Specifically, we implement two simple approaches: moving average (MA) and exponentially weighted moving average (EWMA), which mirror their univariate volatility counterparts.

Let \( x_t = (x_{1,t}, \dots, x_{N,t})^\top \) denote the vector of monetary positions, and let \( w_t = x_t / \sum_{i=1}^N x_{i,t} \) be the corresponding vector of portfolio weights. In this setting, we adopt both a parametric approach—assuming normally distributed returns—and a semi-empirical method based on standardized innovations.

VaR is computed as:
\[
\text{VaR}_t = z_\alpha \cdot \sqrt{w_t^\top \Sigma_t w_t},
\]
where \( \Sigma_t \) is the time-varying covariance matrix estimated by the model.

Expected Shortfall is computed either using the analytical formula under the normal distribution or with the semi-empirical method used for univariate volatility models.

These models are widely used and particularly popular in risk management due to their ability to reflect changes in market structure over time. The semi-empirical variant, using past standardized innovations, is often more accurate than the parametric one, which tends to underestimate risk under the assumption of normally distributed returns.

However, modeling the full variance-covariance (VCV) matrix \( \Sigma_t \) is more delicate than modeling individual volatilities. The presence of unstable or poorly conditioned weights can significantly distort risk measures, making this approach more sensitive to estimation noise. This adds considerable computational cost—especially for large portfolios.

More advanced multivariate models, such as DCC-GARCH or VEC(1,1), are not included due to limited support in Python and their increased estimation complexity.


\section{Factor Models}

We implement the Sharpe model and Fama-French three factor models to estimate portfolio VaR and ES. These models use asset betas with respect to systematic risk factors, allowing us to infer portfolio-level risk from exposures to common sources of variation.

For the Sharpe Model, we assume that each asset’s return $r_i$ follows the linear factor structure:
\[
r_i = \alpha_i + \beta_i r_m + \varepsilon_i,
\]
where $r_m$ is the return on the market portfolio, $\beta_i$ is the asset's sensitivity to the market factor, and $\varepsilon_i$ is a zero-mean idiosyncratic shock uncorrelated with $r_m$ and other residuals. The market return $r_m$ has variance $\sigma_m^2$, and $\mathrm{Var}(\varepsilon_i) = \sigma_{\varepsilon_i}^2$.

The total portfolio variance is:
\[
\sigma_p^2 = \left(\sum_{i=1}^N w_i \beta_i\right)^2 \sigma_m^2 + \sum_{i=1}^N w_i^2 \sigma_{\varepsilon_i}^2.
\]

VaR and ES are computed as follows under the normality assumption:
\[
\text{VaR}_{t,\alpha} = z_\alpha \cdot \sigma_p, \qquad
\text{ES}_{t,\alpha} = \sigma_p \cdot \frac{\phi(z_\alpha)}{1 - \alpha}.
\]

We also extend the previous model by considering three risk factors: market excess return (Mkt-RF), size (SMB), and value (HML). Each asset’s excess return is modeled as:
\[
r_i - r_f = \alpha_i + \beta_{i,1} \cdot \text{Mkt-RF} + \beta_{i,2} \cdot \text{SMB} + \beta_{i,3} \cdot \text{HML} + \varepsilon_i,
\]
where $r_f$ is the risk-free rate, and $\varepsilon_i$ is the idiosyncratic error. We estimate factor loadings $\beta_{i,k}$ through OLS regression.

Let $B$ be the $N \times 3$ matrix of estimated factor loadings, $\Sigma_f$ the $3 \times 3$ sample covariance matrix of the factors, and $\Sigma_\varepsilon$ the diagonal matrix of residual variances $\sigma_{\varepsilon_i}^2$. The total portfolio variance is:
\[
\sigma_p^2 = w^\top (B \Sigma_f B^\top + \Sigma_\varepsilon) w.
\]

Once the portfolio variance is known, we compute VaR and ES as we did for the single factor model.

Factor models are particularly powerful for dimensionality reduction in large portfolios, as they allow one to explain risk using a few systematic sources rather than modeling each asset directly. This is intuitive in finance, where most risk arises from exposure to common factors like the overall market or macroeconomic themes.

However, this approach requires additional data—specifically factor returns and proper asset mapping. The interpretation of portfolio risk also shifts: factor VaR can be viewed as the risk equivalent of holding a leveraged position in the market or factors, rather than in the assets themselves. Like other models assuming normality, it tends to underestimate tail risk.


\section{Analytic VaR}

We compute several Analytic VaR measures using the previously defined monetary position vector \( x_t = (x_{1,t}, \dots, x_{N,t})^\top \) and the covariance matrix \( \Sigma \). 

The quantile \( z_\alpha \) corresponds to the desired tail probability level and is always derived from the standard normal distribution, as the underlying assumption is that asset returns are normally distributed.

The asset-normal parametric portfolio VaR is defined as:
\[
  \text{VaR}_t = z_\alpha \cdot \sqrt{x_t^\top \Sigma x_t} \cdot \sqrt{h}.
\]

Assuming all asset returns are perfectly correlated (\( \rho = 1 \)) and there are no short positions, the undiversified portfolio VaR is simply the sum of the individual asset VaRs:
\[
  \text{UVaR}_t = \sum_{i=1}^N \text{VaR}_{i,t}.
\]

The marginal VaR, representing the sensitivity of portfolio VaR to a small change in position \( x_{i,t} \), is given by:
\[
  \Delta \text{VaR}_{i,t} = \text{VaR}_t \cdot \frac{(\Sigma x_t)_i}{x_t^\top \Sigma x_t}.
\]
It measures the change in total VaR from adding one extra unit of asset \( i \).

The component VaR is:
\[
  \text{CVaR}_{i,t} = x_{i,t} \cdot \Delta \text{VaR}_{i,t},
\]
which captures the absolute contribution of asset \( i \) to total portfolio VaR.

The relative component VaR is:
\[
  \text{RCVaR}_{i,t} = \frac{\text{CVaR}_{i,t}}{\text{VaR}_t},
\]
interpreted as the percentage share of total VaR attributable to asset \( i \).

For a vector \( a \) representing changes in the portfolio allocation, the incremental VaR is:
\[
  \text{IVaR}_t = \Delta \text{VaR}_t^\top \cdot a,
\]
which estimates the change in total VaR resulting from shifting the portfolio by \( a \).

Each of these VaR metrics can be complemented with a corresponding Expected Shortfall (ES), computed under the normality assumption.

Since most of these portfolio risk metrics lack closed-form expressions under more realistic distributions, we are effectively constrained to the normal assumption—despite its well-known shortcomings in capturing heavy tails and asymmetry in financial returns.


\section{Simulation Methods}

Simulation methods are powerful tools for computing VaR and ES in portfolios that include complex instruments such as options, which exhibit non-linear payoffs.

We implement a one-day parametric Monte Carlo simulation approach under the assumption of multivariate normality. Future returns are simulated as:
\[
\mu = \mathbb{E}[r_t], \quad \Sigma = \operatorname{Cov}(r_t), \quad L = \operatorname{Cholesky}(\Sigma),
\]
\[
r^{(i)} = \mu + L z^{(i)}, \quad z^{(i)} \sim \mathcal{N}(0, I),
\]
\[
S^{(i)} = S_0 \circ (1 + r^{(i)}),
\]
where \( \circ \) denotes element-wise multiplication.

The simulated portfolio profit and loss is calculated as:
\[
\Delta P^{(i)} = w^\top(S^{(i)} - S_0) + \sum_{j=1}^{N_{\mathrm{opt}}} q_j \left(C(S_j^{(i)}, \tau_j') - C_{j,0}\right),
\]
where \( q_j \) is the number of option contracts, \( C(\cdot) \) is the Black-Scholes option price, and \( \tau_j' = \max(T_j - \tfrac{1}{252}, 0) \) is the adjusted time to maturity.

From the empirical distribution \( \{\Delta P^{(i)}\}_{i=1}^N \), we compute:
\[
\text{VaR}_\alpha = -\widehat{Q}_\alpha(\Delta P), \qquad
\text{ES}_\alpha = -\mathbb{E}[\Delta P \mid \Delta P \le -\text{VaR}_\alpha].
\]

We also implement a multi-day version of Monte Carlo, which forecasts VaR and ES over a longer horizon. This version is limited to pure equity portfolios.

As a non-parametric alternative, we implement Historical Simulation, both with and without replacement. We simulate alternative return paths by re-applying historical return vectors to the current portfolio. For each scenario \( i \), we define:
\[
S^{(i)} = S_0 \circ (1 + r_{\pi(i)}),
\]
where \( r_{\pi(i)} \) is the \( i \)-th sampled return vector, and \( \pi(i) \) is either the chronological index (historical simulation) or a random draw (bootstrap simulation).

The corresponding portfolio P\&L is:
\[
\Delta P^{(i)} = w^\top (S^{(i)} - S_0).
\]

Risk measures are computed as in the parametric case.

This approach is fully non-parametric and captures fat tails and nonlinear dependence in historical returns. When bootstrap is enabled, repeated resampling improves robustness in small-sample settings.

It is important to note that this is a simplified simulation framework: we do not account for interest rate risk, exchange rate risk, or volatility risk. Additionally, the Black-Scholes-Merton pricing model used for options is itself a stylized and limited approximation, especially under non-normal dynamics. While the normality assumption in Monte Carlo is a known limitation, relaxing it would lead to significantly more complex mathematical and computational challenges. Finally, we do not backtest simulation-based methods, as doing so would require a full rolling re-estimation of simulated distributions at each time step—a process that is extremely computationally intensive, and in many practical contexts, intractable.


\section{Backtesting}


Backtesting assesses whether the observed losses are consistent with the VaR predictions by checking the frequency and structure of violations (exceptions). The objective is to test whether the model is correctly calibrated.

We implement three standard tests.

The Kupiec test (unconditional coverage) evaluates if the number of observed exceptions $N$ over $T$ days matches the expected number under the model. If $p$ is the failure probability, then under the null $N$ follows a binomial distribution. The likelihood ratio test statistic is:
\[
\text{LR}_{\text{uc}} = -2 \left\{ \ln\left[(1 - p)^{T - N} p^N \right] - \ln\left[(1 - \hat{p})^{T - N} \hat{p}^N \right] \right\}, \quad \hat{p} = \frac{N}{T}, \quad \text{LR}_{\text{uc}} \sim \chi^2_1.
\]

The Christoffersen test (conditional coverage) tests the independence of exceptions by modeling their dynamics as a first-order Markov chain. It compares the likelihoods of transition counts under the null (independence) and alternative. The statistic is:
\[
\text{LR}_{\text{c}} = -2 (\ln L_0 - \ln L_1), \quad \text{LR}_{\text{c}} \sim \chi^2_1,
\]
where $L_0$ is the likelihood under independence and $L_1$ under the alternative.

The joint test combines both:
\[
\text{LR} = \text{LR}_{\text{uc}} + \text{LR}_{\text{c}} \sim \chi^2_2.
\]

Rejecting the null in any of the tests suggests that the VaR model is either miscalibrated (too many or too few violations) or fails to capture time dependence in risk (e.g., volatility clustering).

Our back-testing framework covers every method except the simulation models. 
In practice, the Kupiec test is often passed by well-calibrated models such as the Student-$t$ parametric model, historical simulation, or EVT. However, the Christoffersen and joint tests are frequently failed by models that ignore time-varying volatility—such as the normal parametric model—since they tend to underestimate clustering in violations. Being statistical tests, these results are also subject to randomness: even a basic normal VaR model may occasionally pass all tests over short, quiet periods or for low-volatility assets. We generally recommend using at least one year of data for meaningful inference. Finally, while the joint test aggregates both components, it is important to examine each test individually: passing one and failing the other may still result in a non-rejection overall, but masks which specific assumption (coverage or independence) is violated.
\end{document}
