\documentclass{article}
\usepackage[utf8]{inputenc}
\usepackage{amsmath}
\usepackage{lmodern}
\usepackage{parskip}

\title{Automated VaR and ES Estimation Package}
\author{}
\date{}

\begin{document}

\maketitle

We present a Python package for automated Value-at-Risk (VaR) and Expected Shortfall (ES) estimation. It covers core models, portfolio analytics, simulations, and a natural language interface for interpretability. Designed for equity portfolios, the tool is efficient, modular, and easy to extend.

The project was developed collaboratively, with each member contributing distinct components while working together on structure, testing, and documentation.

\textbf{Alessandro Dodon} implemented all volatility models (ARCH/GARCH family), basic VaR methods (historical, normal, Student-t), and the Extreme Value Theory module. He also handled multivariate correlation models and portfolio-level risk decomposition.

\textbf{Marco Gasparetti} worked on simulation methods, including Monte Carlo (parametric and historical), multi-day forecasting, and options-based PL modeling.

\textbf{Niccolò Lecce} integrated a large language model interface for result interpretation, aiming to make outputs more accessible to non-technical users.

Factor models (Sharpe, Fama-French), testing routines, web data ingestion, and interface design were developed jointly by Marco and Niccolò, with group-wide coordination across all tasks.

Backtesting routines were implemented by Alessandro to assess the reliability of VaR models under empirical conditions.

Although tasks were divided, all decisions were made together and every module was reviewed by the full team to ensure consistency and coherence.

\end{document}
